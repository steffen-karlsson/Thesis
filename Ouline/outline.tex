\documentclass[12pt,a4paper]{article}
\usepackage[T1]{fontenc}
\usepackage[utf8]{inputenc}
\usepackage{amsmath}
\usepackage{amsfonts}
\usepackage{amssymb}
\usepackage{listings}
\usepackage{graphicx}
\usepackage{booktabs}
\usepackage{multirow}
\usepackage{listings}
\usepackage{subfigure}
\usepackage[table]{xcolor}
\usepackage{hyperref}

\newcommand{\folge}[1]{\left \lbrace #1 \right \rbrace }
\author{ckh340 Steffen Karlsson}
\title{Master Thesis - Outline \\ \small{rev. 2}}

\definecolor{mygreen}{rgb}{0,0.6,0}
\definecolor{mygray}{rgb}{0.5,0.5,0.5}

\lstset{ %
  basicstyle=\footnotesize,        % the size of the fonts that are used for the code
  breaklines=true,                 % sets automatic line breaking
  captionpos=b,                    % sets the caption-position to bottom
  commentstyle=\color{mygreen},    % comment style
  escapeinside={\%*}{*)},          % if you want to add LaTeX within your code
  frame=none,                    % adds a frame around the code
  keywordstyle=\color{blue},       % keyword style
  language=Java,                 % the language of the code
  numbers=none,                    % where to put the line-numbers; possible values are (none, left, right)
  numbersep=5pt,                   % how far the line-numbers are from the code
  numberstyle=\tiny\color{mygray}, % the style that is used for the line-numbers
  rulecolor=\color{black},         % if not set, the frame-color may be changed on line-breaks within not-black text (e.g. comments (green here))
  stepnumber=2,                    % the step between two line-numbers. If it's 1, each line will be numbered
  linewidth=12cm,
  showspaces=false,
  tabsize=2,                       % sets default tabsize to 2 spaces
  title=\lstname,                   % show the filename of files included with \lstinputlisting; also try caption instead of title
  showstringspaces=false,
}

\begin{document}
\maketitle

\subsection*{BigFS - A semantics aware big data file system}

A progressively\footnote{By the escalation of data, both amount and volume.} increasing challenge in high performance and scientific computing (HPC) is how to store the enormous amount of data on disk, efficiently. Nowadays widely used integrated file system solutions, such as Hadoop\footnote{\url{https://hadoop.apache.org/}}, are developed under other circumstances and with another purpose, than what they are used for now. These systems are usually not designed with HPC in mind, which is noticeable. The systems do not address the fundamental problem of hiding massive IO cost, among others.
\newline

The purpose of this project is to first of all to investigate, whether a distributed parallel file system, that efficiently hides latency and reduce IO-cost is achievable. And secondly, implement such system in a sensibly selected language and environment. 
\newline

By using, what will be defined as, computational filters on incoming data streams and thereby extending the computation time with a fraction, it is presumable possible to hide the latency, streamline the IO operations and thereby reduce the massive cost, when the data eventually need to be fetched and evaluated, computationally. 

This can be archived by adding a meta-data dimension to the file system, in terms of \textit{e.g.} a HDF5\footnote{\url{http://www.hdfgroup.org/HDF5/}} format file. In addition could the computational filters be defined \textit{e.g.} as part of the configuration to the system.

\end{document}\grid
