A variety of different examples of how to utilize and program BDAE efficiently are provided as part of the installation of the complete framework (a guide is available in appendix \ref{app:installation}), which has been tested and verified throughout the development process. 

Furthermore, a range of the most relevant scenarios are additionally tested and validated with the unit test framework\cite{PageUnitTest} in Python. The following chapter is dedicated to describing the unit tests.

\section{SOFA}
\CodeName was carefully tested as part of the BDAE unit tests, but not as a standalone framework, since it is hard to test alone as it is intended to be a subset of a larger system built on top using the available API. 

The tests of BDAE are utilizing a majority of the available crucial API call\footnote{Not the straightforward \texttt{get} method calls.} of \CodeName and are considered comprehensive enough for now, as BDAE would not functioning correctly or at all if \CodeName did not operate correctly.

\section{BDAE}
The unit test cases are implemented using and thus testing the provided BDAE templates described in section \ref{sec:templates} using the adequate user level based BDAE libraries described in section \ref{sec:libraries}.
\newline

\CodeName + BDAE are additionally tested as part of the performance benchmarking described in chapter \ref{chp:benchmark}.

\subsection*{Text}
The test cases for the text-based data covers all three types of BDAE templates explained in section \ref{sec:templates} and the purpose is it test tangible examples.

\testcase{Letter count}{Counting the total number of letters in an arbitrary generated scientific dataset string}{This test are verified using all the three types of text-based templates: \textit{line}, \textit{sentence} or \textit{word}.}{
	\begin{itemize}
		\item \textbf{M}: Repeat count
		\item \textbf{L}: The letter to be repeated $M$ times, an example for $N=4$ and $L=$'a' results in: a aa aaa aaaa.
		\item \textbf{N}: Replication count
	\end{itemize}
}{
\begin{equation*}
	N \cdot \sum_{m=1}^{M+1} m
\end{equation*}
}{All results are tested with $L$ = 'a'.}{
\centering
\begin{tabular}{l p{3cm}}
\specialrule{1.5pt}{2pt}{2pt}
\textbf{Input} & \textbf{Output} \\
\midrule
$N = 1$, $M=1$ & 1 \\ 
$N = 10$, $M=1$ & 55 \\ 
$N = 10$, $M=10$ & 550 \\ 
\specialrule{1.5pt}{2pt}{2pt}
\end{tabular}
}

\testcase{Line count}{Counting the number lines in an arbitrary generated scientific dataset string}{This test are verified using text-based templates: \textit{line}.}{
	\begin{quotation}
		Same as above.
	\end{quotation}
}{
\begin{equation*}
	M
\end{equation*}
}{All results are tested with $L$ = 'a'.}{
\centering
\begin{tabular}{l p{3cm}}
\specialrule{1.5pt}{2pt}{2pt}
\textbf{Input} & \textbf{Output} \\
\midrule
$N = 1$, $M=1$ & 1 \\ 
$N = 10$, $M=1$ & 1 \\ 
$N = 10$, $M=10$ & 10 \\ 
\specialrule{1.5pt}{2pt}{2pt}
\end{tabular}
}

\testcase{Word count}{Counting the number word (defined as text separated by a space in this situation) in an arbitrary generated scientific dataset string}{This test are verified using text-based templates: \textit{word}.}{
	\begin{quotation}
		Same as above.
	\end{quotation}
	\newpage
}{
\begin{equation*}
	N*M
\end{equation*}
}{All results are tested with $L$ = 'a'.}{
\centering
\begin{tabular}{l p{3cm}}
\specialrule{1.5pt}{2pt}{2pt}
\textbf{Input} & \textbf{Output} \\
\midrule
$N = 1$, $M=1$ & 1 \\ 
$N = 10$, $M=1$ & 10 \\ 
$N = 10$, $M=10$ & 100 \\ 
\specialrule{1.5pt}{2pt}{2pt}
\end{tabular}
}
\vspace*{5mm}

\subsection*{Numpy/Bohrium array}
The test cases for the Numpy/Bohrium array template predominantly focuses on the ability to divide the data correctly into the specified semantic blocks and nevertheless calculate the correct results in multiple and various dimensions.

\testcase{Global sum}{Calculating the global sum of the arbitrary generated scientific single and or multidimensional array of fixed data.}{Tested using full arrays of ones, twos etc.}{
	\begin{itemize}
		\item \textbf{M}: Repeat count
		\item \textbf{n}: The number to be repeated in an $M\cdot M$ array.
		\item \textbf{N}: Replication count
	\end{itemize}
}{
\begin{equation*}
	n \cdot M^2 \cdot N
\end{equation*}
}{}{
\centering
\begin{tabular}{l p{3cm}}
\specialrule{1.5pt}{2pt}{2pt}
\textbf{Input} & \textbf{Output} \\
\midrule
$n = 1$, $M=1$, $N=1$ & 1 \\ 
$n = 1$, $M=100$, $N=1$ & 10000 \\ 
$n = 2$, $M=10$, $N=10$ & 2000 \\ 
\specialrule{1.5pt}{2pt}{2pt}
\end{tabular}
}

\testcase{Semantic sum}{Calculating the semantic block sum (assumed to be of dimension $(1 \cdot M)$) of the arbitrary generated scientific single and or multidimensional array of fixed data and verifying that all element results are equal.}{The result is either the sum or -1 if the results are not equal.}{
	\begin{quotation}
		Same as above.
	\end{quotation}
}{
\begin{equation*}
	n \cdot M (\cdot 1)
\end{equation*}
}{Assuming $N=1$ as it is independent and thus does not effect the results.}{
\centering
\begin{tabular}{l p{3cm}}
\specialrule{1.5pt}{2pt}{2pt}
\textbf{Input} & \textbf{Output} \\
\midrule
$n = 1$, $M=1$ & 1 \\ 
$n = 1$, $M=100$ & 100 \\ 
$n = 2$, $M=10$ & 20 \\ 
\specialrule{1.5pt}{2pt}{2pt}
\end{tabular}
}
\newpage

\subsection*{NetCDF}
The two NetCDF test cases concentrate primarily on calculating the global sum of multiple arbitrary sized scientific arrays of fixed data stored jointly together in a NetCDF format, by only loading the data into memory once using the template explained in section \ref{sec:templates}.

\testcase{Dataset sum}{Calculating the sum of a single dataset from a NetCDF format generated with arbitrary scientific single and or multidimensional array of fixed data.}{}{
	\begin{itemize}
		\item \textbf{M}: Repeat count
		\item \textbf{n}: The number to be repeated in an $M\cdot M$ array.
		\item \textbf{N}: Replication count
	\end{itemize}
}{
\begin{equation*}
	n \cdot M^2 \cdot N
\end{equation*}
}{Assuming $N=1$ as it is just an arbitrary multiplication of the base result.}{
\centering
\begin{tabular}{l p{3cm}}
\specialrule{1.5pt}{2pt}{2pt}
\textbf{Input} & \textbf{Output} \\
\midrule
$n = 1$, $M=1$ & 1 \\ 
$n = 2$, $M=100$ & 10000 \\ 
$n = 3$, $M=1000$ & 3000000 \\ 
\specialrule{1.5pt}{2pt}{2pt}
\end{tabular}
}

\testcase{Combined NetCDF sum}{Calculating the total global sum of all datasets generated with arbitrary scientific single and or multidimensional array of fixed data.}{}{
	\begin{quotation}
		Same as above.
	\end{quotation}
}{
\begin{equation*}
	\sum_{\forall n \in \{1,2,3\}} n \cdot M^2 \cdot N
\end{equation*}
}{Assuming $n = \{1,2,3\}$ and $N=1$ for the same reason as for the previous test.}{
\centering
\begin{tabular}{l p{3cm}}
\specialrule{1.5pt}{2pt}{2pt}
\textbf{Input} & \textbf{Output} \\
\midrule
$M=1$ & 6 \\ 
$M=10$ & 600 \\
$M=100$ & 60000 \\ 
\specialrule{1.5pt}{2pt}{2pt}
\end{tabular}
}