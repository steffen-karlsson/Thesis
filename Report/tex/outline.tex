\chapter{Project outline}

A progressively\footnote{By the escalation of data, both amount and volume.} increasing challenge in high performance and scientific computing (HPC) is how to store the enormous amount of data on disk, efficiently. Nowadays widely used integrated file system solutions, such as Hadoop\footnote{\url{https://hadoop.apache.org/}}, are developed under other circumstances, and with another purpose, than what they are used for now. These systems are usually not designed with HPC in mind, which is noticeable. The systems do not address the fundamental problem of hiding massive IO cost, among others.
\newline

The purpose of this project is first of all, to investigate whether a distributed parallel file system that efficiently hides latency and reduce IO-cost is achievable. And secondly, implement such system in a sensibly selected language and environment. 
\newline

By using, what will be defined as, computational filters on incoming data streams and thereby extending the computation time with a fraction, it is presumable possible to hide the latency, streamline the IO operations and thereby reduce the massive cost, when the data eventually need to be fetched and evaluated, computationally. 
\newline

This can be achieved by adding a meta-data dimension to the file system, \textit{e.g.} a HDF5\footnote{\url{http://www.hdfgroup.org/HDF5/}} format file. In addition, the computational filters could be defined \textit{e.g.} as part of the configuration to the system.

\newpage
\section*{Learning goals}
\vspace{3mm}
\begin{itemize}
	\item Investigate and evaluate different file system architectures, in order to optimize performance for big data.	
	\item Analyze one or more approaches, to efficiently pipeline described computations on incoming data.
	\item Define or use existing descriptive configuration language to specify computational filters.
	\item Understand and apply suitable high performance computing and other efficiency measures.
	\item Examine and reflect on performance benchmark results.
\end{itemize}